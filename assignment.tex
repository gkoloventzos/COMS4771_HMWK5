% Use style package recommended by conference.
\documentclass[letterpaper,single,9pt]{article}

\usepackage[all=normal, paragraphs, indent, lists, floats, sections, title]{savetrees}

\usepackage{epsfig}
\usepackage{graphicx}
%\usepackage{subfigure}
\usepackage{subcaption}
\usepackage{verbatim}
\usepackage{amsmath,amssymb}
\usepackage{color}
\usepackage[hyphens]{url}
\usepackage{listings}
\usepackage[normalem]{ulem}
\usepackage{floatrow}
\usepackage{xspace}
\usepackage{amsmath}
\usepackage{enumitem}
\usepackage{booktabs}
\usepackage[utf8]{inputenc}
\usepackage{authblk}
\usepackage{hyperref}
\usepackage{algorithm, algpseudocode}
\usepackage{fancyvrb}
\usepackage{cprotect}
\usepackage{mdframed}
\usepackage{array}
\usepackage{cite}
\usepackage{multirow}
\usepackage{caption}
\usepackage{listings}
\usepackage[usenames,dvipsnames]{xcolor}
\newcommand\independent{\protect\mathpalette{\protect\independenT}{\perp}}
\def\independenT#1#2{\mathrel{\rlap{$#1#2$}\mkern2mu{#1#2}}}
%\renewcommand{\theenumi}{\Alph{enumi}}
\begin{document}
\title{COMS W4771 - Machine Learning 5th Exersice}

  \author{ {\rm Georgios Koloventzos - gk2409} \\ }

\maketitle

\section*{1}
The problem described here is the famous Monty hall problem.
We will analyze the what happens when you choose the first door as
your choise. W.l.o.g. is the same if you choose any of the other 2 doors.
Let assume that the C$x$ event is that the car is behind door $x$.
So $P(Cx) = \frac{1}{3}$. Also let assume that user choosing door $x$ is D$x$.
Also host opening door $x$ is represented H$x$. Also $P(Cx\mid D1) = \frac{1}{3}$.
Also we have the probabilities:
\begin{align*}
P(H3\mid C1,D1) &= \frac{1}{2}\\
P(H3\mid C2,D1) &= 1\\
P(H3\mid C3,D1) &= 0
\end{align*}
Because if the car is in behind door 1 then the host can open any of the other 2.
If car is behind door 2 then he has to open the 3rd door. And it will not open the 
3rd door if the car is behind that door. So if the player select door 1 and the host
opens the 3rd, the probability of winning by switching is:
\begin{align*}
P(C2\mid H3,D1) &= \frac{P(H3\mid C2,D1) P(C2\mid D1)}{P(H3\mid D1)}\\
&= \frac{P(H3\mid C2,D1) P(C2\mid D1)}{P(H3\mid C1,D1) P(C1\mid D1) + P(H3\mid C2,D1) P(C2\mid D1) + P(H3\mid C3,D1) P(C3\mid D1)}\\
&= \frac{P(H3\mid C2,D1)}{P(H3\mid C1,D1) + P(H3\mid C2,D1) + P(H3\mid C3,D1)}\\
&= \frac{1}{\frac{1}{2} + 1 + 0} = \frac{2}{3}
\end{align*}
So it is better to change doors.

\section*{2}
The propability $p(x_1,\dotsc,x_5) = \prod_{i=i}^5 p(x_i\mid \pi_i)$.
So for our graph we have:
\begin{align*}
p(x_1,\dotsc,x_5) = p(x_1)p(x_3)p(x_2\mid x_1)p(x_4\mid x_{1},x_3)p(x_5\mid x_{2},x_4)
\end{align*}
\begin{enumerate}
\item False. As no node is shaded the ball can go from $x_2$ to $x_1$ and reach $x_4$ (2 effects handout 15 p.12)
\item 
\item
\item
\item
\item
\item
\item True. The ball will bounce at $x_5$ and also at $x_4$. So $x_2 \independent x_3$. 
\item False. Because now the ball passes from $x_5$
\item False. Because now the ball passes from $x_5$
\end{enumerate}

\section*{3}
In this pictures we can see the moralized edge(red) and the one for
triangulation(green). As the onle 2 nodes with same children are B1 and 
B4, are the only edge that we add in the moralized phase. The green edge
is added in the triangulation phase as the B1,B2,B3,B4 are the only clique.
\begin{figure}[ht]
  \begin{subfigure}[b]{0.5\linewidth}
    \centering
    \includegraphics[width=0.75\linewidth]{figures/ml5.jpg}
    \caption{Moralized/Triangulated graph}
    \vspace{4ex}
  \end{subfigure}%%
  \begin{subfigure}[b]{0.5\linewidth}
    \centering
    \includegraphics[width=0.75\linewidth]{figures/ml5triang.jpg}
    \caption{Clique graph}
    \vspace{4ex}
  \end{subfigure}
\end{figure}
\clearpage
\begin{figure}[ht]
    \centering
    \includegraphics[width=0.75\textwidth]{figures/ml5JT.jpg}
    \caption{Junction Tree}
\end{figure}

\section*{4}
\begin{figure}[ht]
    \centering
    \includegraphics[width=0.5\textwidth]{figures/ml4.jpg}
    \caption{Junction Tree}
\end{figure}

\section*{5}
For this problem we will use Hidden Markov Model to find what is the most probable
state of Super Mario. We have $T=5$ observations. $N=2$ number of states $Q=\{Happy(H), Angry(A)\}$,
$M=4$ observations symbols with $V=\{smile(s), frown(f), laugh(l), yell(y)\}$.
The state transision probabilies are depicting in table ~\ref{tab:a}. The observations
we have is $O = \{s, y, f, f, l\}$.
\begin{table}[ht]
 \centering
  \begin{tabular}{ | c || c | c | }
    \hline
     & H & A \\ \hline
    H & 0.8 & 0.2 \\ \hline
    A & 0.2 & 0.8 \\ \hline
  \end{tabular}
    \caption{Transition Probabilities}\label{tab:a}
\end{table}
The emmision probabilties can be found in table ~\ref{tab:b}
\begin{table}[ht]
 \centering
  \begin{tabular}{ | c || c | c | c | c | }
    \hline
     & s & y & f & l \\ \hline
    H & 0.4 & 0.1 & 0.3 & 0.2 \\ \hline
    A & 0.1 & 0.4 & 0.2 & 0.3 \\ \hline
  \end{tabular}
    \caption{Emission probabilities}\label{tab:b}
\end{table}
The initial state probability is $\pi = [ 1 0 ]$ as the problem statement tells us that he is happy.
In order to find the most probable state sequence we will calculate the probability of each state stransition
with the following formula.\\
\\
\begin{align*}
P(X) = \pi_{x_1}b_{x_1}(O_1)a_{{x_1},{x_2}}b_{x_2}(O_2)a_{{x_2},{x_3}}b_{x_3}(O_3)a_{{x_3},{x_4}}b_{x_4}(O_4)a_{{x_4},{x_5}}b_{x_5}
\end{align*}
where $X = \{x_{1}, x_{2}, x_{3}, x_{4}, x_{5}\}$ is a sequence of states and $O_{1},O_{2},O_{3},O_{4},O_{5}$ 
are the corresponding observations. Moreover $\pi_{x_1}$ is the initial probability which is equal 
to 1 as we know Super Mario is happy. Futhermore $b_x(y)$ is the probability of observation y at state x (Table ~\ref{tab:b}) and
$a_{i,j}$ is the probability of the transition from state i to state j (Table ~\ref{tab:a}).
\begin{table}[ht]
 \centering
  \begin{tabular}{ | c || c | c | }
    \hline
    State & Probability & Normalized Probability \\ \hline
    HHHHH & 0.000098304 & 0.082139037 \\ \hline
    HHHHA & 0.000016384 & 0.013689840 \\ \hline
    HHHAH & 0.000024576 & 0.020534759 \\ \hline
    HHHAA & 0.000065536 & 0.054759358 \\ \hline
    HHAHH & 0.000024576 & 0.020534759 \\ \hline
    HHAHA & 0.000004096 & 0.003422460 \\ \hline
    HHAAH & 0.000098304 & 0.082139037 \\ \hline
    HHAAA & 0.000262144 & 0.219037433 \\ \hline
    HAHHH & 0.000009216 & 0.007700535 \\ \hline
    HAHHA & 0.000001536 & 0.001283422 \\ \hline
    HAHAH & 0.000002304 & 0.001925134 \\ \hline
    HAHAA & 0.000006144 & 0.005133690 \\ \hline
    HAAHH & 0.000036864 & 0.030802139 \\ \hline
    HAAHA & 0.000006144 & 0.005133690 \\ \hline
    HAAAH & 0.000006144 & 0.123208556 \\ \hline
    HAAAA & 0.000393216 & 0.328556150 \\ \hline
  \end{tabular}
    \caption{State Sequence Probabilities}\label{tab:c}
\end{table}

Having the normalized probabilities we will add the probabilities of state Happy for every
day and find out what is the most probable state with the HMM probabilities.

\begin{table}[ht]
 \centering
  \begin{tabular}{ | c || c | c | c | c | c |}
    \hline
    State Probability & Day 1 & Day 2 & Day 3 & Day4 & Day 5 \\ \hline
    H & 1 & 0.496256683 & 0.187165775 & 0.164705882 & 0.368983956 \\ \hline
    A & 0 & 0.503743317 & 0.812834225 & 0.835294118 & 0.631016044 \\ \hline
  \end{tabular}
    \caption{HMM probabilities}\label{tab:hmm}
\end{table}

From table ~\ref{tab:hmm} we observe that the most likely state sequence is \\
$\{Happy, Angry, Angry, Angry, Angry\}$ probably because the princess is in another castle!


\end{document}

